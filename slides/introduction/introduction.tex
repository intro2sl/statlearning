\documentclass[12pt]{beamer}\usepackage[]{graphicx}\usepackage[]{color}
%% maxwidth is the original width if it is less than linewidth
%% otherwise use linewidth (to make sure the graphics do not exceed the margin)
\makeatletter
\def\maxwidth{ %
  \ifdim\Gin@nat@width>\linewidth
    \linewidth
  \else
    \Gin@nat@width
  \fi
}
\makeatother

\definecolor{fgcolor}{rgb}{0.345, 0.345, 0.345}
\newcommand{\hlnum}[1]{\textcolor[rgb]{0.686,0.059,0.569}{#1}}%
\newcommand{\hlstr}[1]{\textcolor[rgb]{0.192,0.494,0.8}{#1}}%
\newcommand{\hlcom}[1]{\textcolor[rgb]{0.678,0.584,0.686}{\textit{#1}}}%
\newcommand{\hlopt}[1]{\textcolor[rgb]{0,0,0}{#1}}%
\newcommand{\hlstd}[1]{\textcolor[rgb]{0.345,0.345,0.345}{#1}}%
\newcommand{\hlkwa}[1]{\textcolor[rgb]{0.161,0.373,0.58}{\textbf{#1}}}%
\newcommand{\hlkwb}[1]{\textcolor[rgb]{0.69,0.353,0.396}{#1}}%
\newcommand{\hlkwc}[1]{\textcolor[rgb]{0.333,0.667,0.333}{#1}}%
\newcommand{\hlkwd}[1]{\textcolor[rgb]{0.737,0.353,0.396}{\textbf{#1}}}%
\let\hlipl\hlkwb

\usepackage{framed}
\makeatletter
\newenvironment{kframe}{%
 \def\at@end@of@kframe{}%
 \ifinner\ifhmode%
  \def\at@end@of@kframe{\end{minipage}}%
  \begin{minipage}{\columnwidth}%
 \fi\fi%
 \def\FrameCommand##1{\hskip\@totalleftmargin \hskip-\fboxsep
 \colorbox{shadecolor}{##1}\hskip-\fboxsep
     % There is no \\@totalrightmargin, so:
     \hskip-\linewidth \hskip-\@totalleftmargin \hskip\columnwidth}%
 \MakeFramed {\advance\hsize-\width
   \@totalleftmargin\z@ \linewidth\hsize
   \@setminipage}}%
 {\par\unskip\endMakeFramed%
 \at@end@of@kframe}
\makeatother

\definecolor{shadecolor}{rgb}{.97, .97, .97}
\definecolor{messagecolor}{rgb}{0, 0, 0}
\definecolor{warningcolor}{rgb}{1, 0, 1}
\definecolor{errorcolor}{rgb}{1, 0, 0}
\newenvironment{knitrout}{}{} % an empty environment to be redefined in TeX

\usepackage{alltt}
\usepackage{tikz}

% make it pretty
% get rid of junk
\usetheme{default}
\usefonttheme[onlymath]{serif}
\beamertemplatenavigationsymbolsempty

% define a bunch of colors
\definecolor{offwhite}{RGB}{255,250,240}
\definecolor{gray}{RGB}{155,155,155}
\definecolor{foreground}{RGB}{80,80,80}
\definecolor{background}{RGB}{255,255,255}
%\definecolor{title}{RGB}{255,199,0}
\definecolor{title}{RGB}{89,132,212}
%\definecolor{subtitle}{RGB}{89,132,212}
\definecolor{subtitle}{RGB}{255,199,0}
\definecolor{hilit}{RGB}{248,117,79}
\definecolor{vhilit}{RGB}{255,111,207}
\definecolor{lolit}{RGB}{200,200,200}
\definecolor{lit}{RGB}{255,199,0}
\definecolor{mdlit}{RGB}{89,132,212}
\definecolor{link}{RGB}{248,117,79}

% a few color macros
\newcommand{\hilit}{\color{hilit}}
\newcommand{\vhilit}{\color{vhilit}}
\newcommand{\lit}{\color{lit}}
\newcommand{\mdlit}{\color{mdlit}}
\newcommand{\lolit}{\color{lolit}}

% use those colors
\setbeamercolor{titlelike}{fg=title}
\setbeamercolor{subtitle}{fg=subtitle}
\setbeamercolor{frametitle}{fg=gray}
%\setbeamercolor{structure}{fg=subtitle}
\setbeamercolor{structure}{fg=title}
\setbeamercolor{institute}{fg=lolit}
\setbeamercolor{normal text}{fg=foreground,bg=background}
\setbeamertemplate{itemize subitem}{{\textendash}}
\setbeamerfont{itemize/enumerate subbody}{size=\small}
\setbeamerfont{itemize/enumerate subitem}{size=\small}

% center title of slides
\setbeamertemplate{blocks}[rounded]
\setbeamertemplate{frametitle}[default][center]

% page number
\setbeamerfont{page number in foot}{size=\footnotesize}
\setbeamertemplate{footline}[frame number]

% default link color
\hypersetup{colorlinks, urlcolor={link}}

% a few macros
\newcommand{\code}[1]{\texttt{#1}}
\newcommand{\hicode}[1]{{\hilit \texttt{#1}}}
\newcommand{\locode}[1]{{\lolit \texttt{#1}}}
\newcommand{\bb}[1]{\begin{block}{#1}}
\newcommand{\eb}{\end{block}}
\newcommand{\bi}{\begin{itemize}}
\newcommand{\bbi}{\vspace{4pt} \begin{itemize} \itemsep8pt}
\newcommand{\ei}{\end{itemize}}
\newcommand{\bv}{\begin{verbatim}}
\newcommand{\ev}{\end{verbatim}}
\newcommand{\ig}{\includegraphics}
\newcommand{\subt}[1]{{\footnotesize \color{subtitle} {#1}}}
\newcommand{\ttsm}{\tt \small}
\newcommand{\ttfn}{\tt \footnotesize}
\newcommand{\figh}[2]{\centerline{\includegraphics[height=#2\textheight]{#1}}}
\newcommand{\figw}[2]{\centerline{\includegraphics[width=#2\textwidth]{#1}}}



%------------------------------------------------

\title{Introduction}
\subtitle{Intro 2 Statistical Learning}
\author{\href{http://www.gastonsanchez.com}{Gaston Sanchez}}
\institute{\href{https://creativecommons.org/licenses/by-sa/4.0/}{\tt \scriptsize \color{foreground} CC BY-SA 4.0}}
\date{}
\IfFileExists{upquote.sty}{\usepackage{upquote}}{}
\begin{document}



% no page number in first slide
{
  \setbeamertemplate{footline}{} 
  \frame{\titlepage} 
}

%------------------------------------------------

\begin{frame}
\begin{center}
\Huge{{\lolit An introduction to} 
\\ {\mdlit Predictive Modeling} \\ {\lit and Statistical Learning}}
\end{center}
\end{frame}

%------------------------------------------------

\begin{frame}
\frametitle{Statistical Learning Branches}

{\large
\begin{center}
 \begin{tabular}{l c c}
  \hline
  {\mdlit \textbf{Statistics}} &  & {\lit \textbf{Machine Learning}} \\
  \hline
   & \\
  Predictive   & &  Supervised  \\
  methods      & &  learning \\
   & & \\
  \hline
   & & \\
  Descriptive & &  Unsupervised \\
  methods     & &  learning \\
   & & \\
  \hline
 \end{tabular}
\end{center}
}

\end{frame}

%------------------------------------------------

\begin{frame}
\begin{center}
\ig[width=11cm]{images/four_corners.pdf}
\end{center}
\end{frame}

%------------------------------------------------

\begin{frame}
\frametitle{A word of caution}

Sometimes there might not be a clear distinction between supervised and 
unsupervised learning. Often, a given method mixes both types of approaches.

\end{frame}

%------------------------------------------------

\begin{frame}
\begin{center}
\Huge{\hilit{Data Analysis Cycle \\ (DAC)}}
\end{center}
\end{frame}

%------------------------------------------------

\begin{frame}
\frametitle{Cycle of Data Anlaysis Projects}
\begin{center}
\ig[width=8cm]{images/cycle1.pdf}
\end{center}
\end{frame}

%------------------------------------------------

\begin{frame}
\frametitle{Cycle of Data Anlaysis Projects}
\begin{center}
\ig[width=8cm]{images/cycle2.pdf}
\end{center}
\end{frame}

%------------------------------------------------

\begin{frame}
\frametitle{Cycle of Data Anlaysis Projects}
\begin{center}
\ig[width=8cm]{images/cycle3.pdf}
\end{center}
\end{frame}

%------------------------------------------------

\begin{frame}
\frametitle{Cycle of Data Anlaysis Projects}
\begin{center}
\ig[width=8cm]{images/cycle4.pdf}
\end{center}
\end{frame}

%------------------------------------------------

\begin{frame}
\begin{center}
\ig[width=11cm]{images/data-by-the-numbers.png}

{\tiny \url{http://www.phdcomics.com/comics/archive.php/archive/tellafriend.php?comicid=462}}
\end{center}
\end{frame}

%------------------------------------------------

\begin{frame}
\frametitle{Data Preparation}
\begin{center}
\ig[width=4cm]{images/databynumbers1.pdf}
\end{center}
\end{frame}

%------------------------------------------------

\begin{frame}
\frametitle{Core Data Analysis}
\begin{center}
\ig[width=4cm]{images/databynumbers2.pdf}
\end{center}
\end{frame}

%------------------------------------------------

\begin{frame}
\frametitle{Reporting}
\begin{center}
\ig[width=4cm]{images/databynumbers3.pdf}
\end{center}
\end{frame}

%------------------------------------------------

\begin{frame}
\frametitle{Communication}
\begin{center}
\ig[width=4cm]{images/databynumbers4.pdf}
\end{center}
\end{frame}

%------------------------------------------------

\begin{frame}
\frametitle{Keep in mind}
\begin{center}
\ig[width=11cm]{images/dac0.pdf}
\end{center}
\end{frame}

%------------------------------------------------

\begin{frame}
\frametitle{(Some) Major Data Analysis Tasks}

\bbi
  \item \textbf{Visualization}: to facilitate human discovery
  \item \textbf{Summarizing}: describing information
  \item \textbf{Deviation Detection}: finding changes
  \item \textbf{Profiling}: finding relevant characteristics of a group of individuals
  \item \textbf{Associations}: finding relationships, e.g. A \& B \& C occur frequently
  \item \textbf{Clustering}: finding groups in data
  \pause
  \item {\large \hilit \textbf{Prediction}}
\ei

\end{frame}

%------------------------------------------------

\begin{frame}
\frametitle{Keep in mind}
\begin{center}
\ig[width=11cm]{images/dac1.pdf}
\end{center}
\end{frame}

%------------------------------------------------

\begin{frame}
\frametitle{Keep in mind}
\begin{center}
\ig[width=11cm]{images/dac2.pdf}
\end{center}
\end{frame}

%------------------------------------------------

\begin{frame}
\frametitle{Official Title}
\begin{center}
\Huge{\hilit Modern Statistical Prediction?}
\end{center}
\end{frame}

%------------------------------------------------

\begin{frame}
\frametitle{Modern Statistical Prediction?}

\bbi
  \item Statistical Prediction is not a new task
  \item Predictive applications (least squares) date back to 18th-19th century 
  (Andrien-Marie Legendre -vs- Carl Friedrich Gauss)
  \item Regression framework originated at the beginning of 20th century
  (Francis Galton, Karl Pearson, Udny Yule)
  \item Classification framework originated around the 1930s
  (Ronald Fisher, P.C. Mahalanobis, B.L. Welch)
\ei

\end{frame}

%------------------------------------------------

\begin{frame}
\begin{center}
{\large \mdlit So where does the ``modern'' part come from?}
\end{center}
\end{frame}

%------------------------------------------------

\begin{frame}
\frametitle{Modern Statistical Prediction}

\bb{So where does the ``modern'' part come from?}
\bbi
  \item Model concept
  \item Data Sets
  \item Fields of Application
  \item Computing Tools
  \item Mathematical/Algorithmic Tweaks
  \item Predictive performance assessment 
  \item Modeling Pipeline
\ei
\eb

\end{frame}

%------------------------------------------------

\begin{frame}
\frametitle{Concept of a Model}

\bbi
  \item Term ``model'' appeared in the 1930s (econometric models).
  \item The concept of model has not remained static.
  \item Way of thinking about what a model varies across disciplines.
  \item Even within the same community, there may be different ideas of ``model''.
\ei

\end{frame}

%------------------------------------------------

\begin{frame}
\frametitle{Concept of a Model}

\bbi
  \item Suppose we observe a response $Y$ 
  \item We also observe $p$ different predictors, $X_1, X_2, \dots, X_p$
  \item We assume $Y$ is related with $[X_1, \dots, X_p]$
  \item The relationship can be written in a general form as \\
  $$
  Y = f(X_1, X_2, \dots, X_p) + \epsilon
  $$
\ei

\end{frame}

%------------------------------------------------

\begin{frame}
\frametitle{Concept of a Model}

{\large
$$
Y = f(X_1, X_2, \dots, X_p) + \epsilon
$$
}

\bbi
  \item $f()$ represents the systematic information---the {\mdlit \textit{signal}}---that 
  the predictors provide about $Y$
  \item $\epsilon$ represents an \textit{error} term---the {\mdlit \textit{noise}}---that 
  is a catch-all for what we miss with the model
\ei

\begin{center}
{\lit What kind of $f()$?}
\end{center}

\end{frame}

%------------------------------------------------

\begin{frame}
\frametitle{What kind of $f()$?}

$$
Y = f(X_1, X_2, \dots, X_p) + \epsilon
$$

\bbi
  \item In ``classic'' statistics, $f()$ takes the form of a function 
  {\lolit (with parameters to be estimated)}
  \item Within statistical learning, $f()$ is more open-ended
  \item It can also take the form of an algorithm
  \item Sometimes $f()$ is a \textit{black box}
\ei

\end{frame}

%------------------------------------------------

\begin{frame}
\begin{center}
{\lolit So where does the ``modern'' part come from?}

\Large{\mdlit Data Sets}
\end{center}
\end{frame}

%------------------------------------------------

\begin{frame}
\frametitle{The three V's of Data}
\begin{center}
\ig[width=7cm]{images/vol-var-vel.pdf}

{\footnotesize {\lolit (3Vs from conversation with Prof. David Ackerly)}}
\end{center}
\end{frame}

%------------------------------------------------

\begin{frame}
\frametitle{Modern Statistical Prediction?}

\bb{The three V's}
\bbi
  \item \textbf{Volume}: larger data sets with reduced storage cost.
  \item \textbf{Velocity}: increasing rate at which data is produced/recorded.
  \item \textbf{Variety}: new types of data, more diverse/complex.
\ei
\eb

\end{frame}

%------------------------------------------------

\begin{frame}
\frametitle{Volume}
\begin{center}
\ig[width=10cm]{images/volume-servers.pdf}
\end{center}
\end{frame}

%------------------------------------------------

\begin{frame}
\frametitle{Velocity}
\begin{center}
\ig[width=10cm]{images/velocity-binary.pdf}
\end{center}
\end{frame}

%------------------------------------------------

\begin{frame}
\frametitle{Variety}
\begin{center}
\ig[width=10cm]{images/variety-science.pdf}
\end{center}
\end{frame}

%------------------------------------------------

\begin{frame}
\frametitle{Variety}
\begin{center}
\ig[width=10cm]{images/variety-med.pdf}
\end{center}
\end{frame}

%------------------------------------------------

\begin{frame}
\frametitle{Variety}
\begin{center}
\ig[width=10cm]{images/variety-noaa.pdf}
\end{center}
\end{frame}

%------------------------------------------------

\begin{frame}
\frametitle{Variety}
\begin{center}
\ig[width=10cm]{images/variety-social.pdf}
\end{center}
\end{frame}

%------------------------------------------------

\begin{frame}
\begin{center}
{\lolit So where does the ``modern'' part come from?}

\Large{\mdlit Fields of Application}
\end{center}
\end{frame}

%------------------------------------------------

\begin{frame}
\frametitle{Fields of Application Example}

For instance, consider the history of PLS Regression \\
{\scriptsize \lolit (I'll talk about this with more detail when we study PLSR)}

\bi
  \item Origins in mid-1960s with Herman Wold
  \item As a side-project Wold deviced a series of algorithms based on Least Squares steps
  \item First applications in Psychometrics and Econometrics
  \item Karl Joreskog (Wold's former PhD student) disruption 
  of Structural Equation Models (1970s)
  \item Explosion of applications in Education, Sociology, Psychology
\ei

\end{frame}

%------------------------------------------------

\begin{frame}
\frametitle{Fields of Application Example (cont'd)}

For instance, consider the history of PLS Regression \\
{\scriptsize \lolit (I'll talk about this with more detail when we study PLSR)}

\bi
  \item Inspired by Joreskog's work, Wold's refined his framework 
  \item Extension to multivariate regressions and systems of equations
  \item Herman Wold's framework poorly acknowledged (for various reasons)
  \item Applied to chemometrics in late 1970s
  \item Further adaptations by his son Svante Wold, and Harald Martens
  \item New regression approach via Partial Least Squares
\ei

\end{frame}

%------------------------------------------------

\begin{frame}
\begin{center}
{\lolit So where does the ``modern'' part come from?}

\Large{\mdlit Mathematical/Algorithmic Tweaks}
\end{center}
\end{frame}

%------------------------------------------------

\begin{frame}
\frametitle{Mathematical/Algorithmic tweaks example}

Multiple Linear Regression
$$
Y = \beta_0 + \beta_1 X_1 + \beta_2 X_2 + \dots + \beta_p X_p + \epsilon
$$

In matrix notation
$$
\mathbf{y} = \mathbf{X} \boldsymbol{\beta} + \boldsymbol{\epsilon}
$$

\pause
Predicted model 
$$
\mathbf{\hat{y}} = \mathbf{X} \boldsymbol{\hat{\beta}} = \mathbf{X b}
$$

\end{frame}

%------------------------------------------------

\begin{frame}
\frametitle{Mathematical/Algorithmic tweaks example}

OLS solution given by minimizing the residual sum of squares:
$$
min \quad \sum_{i=1}^{n} \left ( y_i - b_0 - \sum_{j=1}^{p} b_j x_j \right )^2
$$

in vector-matrix notation:
$$
min \quad \| \mathbf{y - Xb} \|^2
$$

\end{frame}

%------------------------------------------------

\begin{frame}
\frametitle{Mathematical/Algorithmic tweaks example}

Assuming that $\mathbf{X}$ is of full column-rank, the OLS solution for
$$
\mathbf{\hat{y}} = \mathbf{X} \boldsymbol{\hat{\beta}} = \mathbf{X b}
$$

is given by:
$$
\mathbf{b} = \mathbf{(X^\mathsf{T} X)^{-1} X^{\mathsf{T}} y}
$$

\end{frame}

%------------------------------------------------

\begin{frame}
\frametitle{Mathematical/Algorithmic tweaks example}

Potential instability---due to multicollinearity---in the OLS solution affecting

{\Large
$$
\mathbf{(X^{\mathsf{T}} X)^{-1}}
$$
}

\end{frame}

%------------------------------------------------

\begin{frame}
\frametitle{Mathematical/Algorithmic tweaks example}

{\mdlit One option}: Find inverse of $\mathbf{(X^{\mathsf{T}} X)}$ by
looking for an orthogonal basis:

{\Large
$$
(\mathbf{X^{\mathsf{T}} X})^{-1} \approx \mathbf{V \Lambda_{*}^{-1} V^\mathsf{T}}
$$
}

\pause
\bigskip
{\lit Another option}: Modify $\mathbf{(X^{\mathsf{T}} X)^{-1}}$ by
adding a small constant $k$ to the diagonal entries of $\mathbf{X^{\mathsf{T}} X}$
before taking the inverse:

{\Large
$$
\mathbf{X^{\mathsf{T}} X} + k \mathbf{I}
$$
}

\end{frame}

%------------------------------------------------

\begin{frame}
\begin{center}
{\lolit So where does the ``modern'' part come from?}

\Large{\mdlit Concept of ``Predictive Modeling''}
\end{center}
\end{frame}

%------------------------------------------------

\begin{frame}
\frametitle{Modeling Goals}

\bb{A statistical model typically aims to}
Provide a certain comprehension of the data and the mechanism
that generated them through a parsimonious representation of 
a random phenomenon.
\eb

\bb{Sometimes also, a statistical model seeks to}
Predict new observations with ``good'' accuracy.
\eb

\end{frame}

%------------------------------------------------

\begin{frame}
\frametitle{Modeling for what?}

\centerline{\mdlit \Large Goal Tradeoff}

\bigskip
\centerline{\Large Understanding \quad -vs- \quad Prediction}

\end{frame}

%------------------------------------------------

\begin{frame}
\frametitle{Introduction}

\bb{Understanding?}
Understand could mean a model of a distribution for a random 
vector but it could also mean a regression model.

\bigskip
From a classic point of view, a model should be simple, and its 
parameters should be interpretable in terms of its domain of 
application (e.g. elasticity, odds-ratio, etc).
\eb

\end{frame}

%------------------------------------------------

\begin{frame}
\frametitle{Paradoxes}

\bb{Paradox 1}
A ``good'' statistical model does not necessarily gives accurate 
predictions (at an individual level). E.g. risk factors in epidemiology.
\eb

\pause

\bb{Paradox 2}
We can predict without understanding
\bi
  \item no need for a theory of consumer to predict marketing target
  \item a model may be just simply an algorithm
\ei
\eb

\end{frame}

%------------------------------------------------

\begin{frame}
\frametitle{Inference}

\bb{Classic Inferential Statistics}
Methodology for extracting information from data and expressing 
the amount of uncertainty in decisions we make.
\eb

\bi
  \item Assume distributions for the data
  \item Inferential aspects
  \item More theory-based
  \item More focused on testing hypotheses
\ei

\end{frame}

%------------------------------------------------

\begin{frame}
\begin{center}
{\lolit So where does the ``modern'' part come from?}

\Large{\mdlit Assessing Predictive Performance}
\end{center}
\end{frame}

%------------------------------------------------

\begin{frame}
\frametitle{Model Performance}

\bb{How do we define what a ``good'' model is?}
\bbi
  \item A model that fits the data well? \\
  {\lolit (e.g. minimize resubstitution error)}
  \item A model with optimal parameters? \\
  {\lolit (e.g. most likely coefficients)}
  \item A model that adequately predicts new (unseen) observations? \\
  {\lolit (e.g. minimize generalization error)}
\ei
\eb

\end{frame}

%------------------------------------------------

\begin{frame}
\frametitle{Predictive Modeling}

{\Large
\begin{quotation}
\noindent The Process of developing a mathematical tool or model that generates an accurate prediction.

\bigskip
{\small \noindent \textit{Kuhn and Johnson, 2013}}

\end{quotation}
}

\end{frame}

%------------------------------------------------

\begin{frame}
\frametitle{Predictive Modeling}

{\Large 
\begin{quotation}
\noindent The art of building and using models that make predictions based on patterns extracted from historical data.

\bigskip
{\small \noindent \textit{Kelleher et al, 2015}}
\end{quotation}
}

\end{frame}

%------------------------------------------------

\begin{frame}
\frametitle{Model Performance}

\bbi
  \item From the predictive modeling standpoint, a ``good'' model is one which gives accurate predictions.
  \item By \textit{predictions} we mean predictions of new data.
  \item Therefore we focus on the generalization ability of the model to predict unobserved data
  \item This involves finding measure(s) of accuracy for predictions.
\ei

\end{frame}

%------------------------------------------------

\begin{frame}
\begin{center}
{\lolit So where does the ``modern'' part come from?}

\Large{\mdlit Modeling Pipeline}
\end{center}
\end{frame}

%------------------------------------------------

\begin{frame}[fragile]
\frametitle{Cycle of DAP and Predictive Modeling}

\bi
  \item Data collection
  \item Data preparation (cleansing, formatting, transformations)
  \bi
    \item Feature selection
    \item Feature extraction
  \ei
  \item Model Building
  \bi
    \item Select modeling techniques
    \item Select validation approach
    \item Find optimal model
  \ei
  \item Evaluation
  \item Deployment (decision making)
\ei

\end{frame}

%------------------------------------------------

\begin{frame}
\frametitle{Predictive Modeling Process}

\bb{Main Considerations}
\begin{enumerate}
  \item What data do you have?
  \item What do you want to predict about the data?
  \item What predictive methods/techniques should you use?
  \item How accurate predictions look like?
  \item What is the predictive performance?
  \item Is there overfitting?
\end{enumerate}
\eb

\end{frame}

%------------------------------------------------

\begin{frame}
\frametitle{Predictive Modeling}

{\large 
\begin{quotation}
\noindent We think that good data analysis depends not only on clear thinking but also on substantive knowledge. Mere numerology will not do, nor is there a good cookbook.

\bigskip
{\small \noindent \textit{David Freedman, 1987}}
\end{quotation}
}

\end{frame}

%------------------------------------------------

\begin{frame}
\frametitle{Terminology (Lebart, 1995)}

\small{
\begin{center}
 \begin{tabular}{l l}
  \hline
   Statistics & Machine Learning \\
  \hline
  Variables & Attributes (fields) \\
  Individuals (objects, observations) & Instances (records, samples) \\
  Predictors (independent) & Input \\  
  Response (dependent) & Output (target) \\
  Model & Machine \\
  Coefficients & Weights \\
  Fit Criteria & Cost function \\
  Estimation & Learning \\
  Prediction & Supervised \\
  Structure & Unsupervised \\
  \hline
 \end{tabular}
\end{center}
}

\end{frame}

%------------------------------------------------

\begin{frame}
\frametitle{Bibliography}

{\footnotesize
\bi
  \item \textbf{Modern Multivariate Statistical Techniques} by Izenman (2008). Springer.
  \item \textbf{Applied Predictive Modeling} by Kuhn and Johnson (2013).
  \item \textbf{Fundamentals for Machine Learning for Predictive Data Analytics} by Kelleher et al (2015).
\ei
}

\end{frame}

%------------------------------------------------

\end{document}
