\documentclass[12pt]{beamer}\usepackage[]{graphicx}\usepackage[]{color}
%% maxwidth is the original width if it is less than linewidth
%% otherwise use linewidth (to make sure the graphics do not exceed the margin)
\makeatletter
\def\maxwidth{ %
  \ifdim\Gin@nat@width>\linewidth
    \linewidth
  \else
    \Gin@nat@width
  \fi
}
\makeatother

\definecolor{fgcolor}{rgb}{0.345, 0.345, 0.345}
\newcommand{\hlnum}[1]{\textcolor[rgb]{0.686,0.059,0.569}{#1}}%
\newcommand{\hlstr}[1]{\textcolor[rgb]{0.192,0.494,0.8}{#1}}%
\newcommand{\hlcom}[1]{\textcolor[rgb]{0.678,0.584,0.686}{\textit{#1}}}%
\newcommand{\hlopt}[1]{\textcolor[rgb]{0,0,0}{#1}}%
\newcommand{\hlstd}[1]{\textcolor[rgb]{0.345,0.345,0.345}{#1}}%
\newcommand{\hlkwa}[1]{\textcolor[rgb]{0.161,0.373,0.58}{\textbf{#1}}}%
\newcommand{\hlkwb}[1]{\textcolor[rgb]{0.69,0.353,0.396}{#1}}%
\newcommand{\hlkwc}[1]{\textcolor[rgb]{0.333,0.667,0.333}{#1}}%
\newcommand{\hlkwd}[1]{\textcolor[rgb]{0.737,0.353,0.396}{\textbf{#1}}}%
\let\hlipl\hlkwb

\usepackage{framed}
\makeatletter
\newenvironment{kframe}{%
 \def\at@end@of@kframe{}%
 \ifinner\ifhmode%
  \def\at@end@of@kframe{\end{minipage}}%
  \begin{minipage}{\columnwidth}%
 \fi\fi%
 \def\FrameCommand##1{\hskip\@totalleftmargin \hskip-\fboxsep
 \colorbox{shadecolor}{##1}\hskip-\fboxsep
     % There is no \\@totalrightmargin, so:
     \hskip-\linewidth \hskip-\@totalleftmargin \hskip\columnwidth}%
 \MakeFramed {\advance\hsize-\width
   \@totalleftmargin\z@ \linewidth\hsize
   \@setminipage}}%
 {\par\unskip\endMakeFramed%
 \at@end@of@kframe}
\makeatother

\definecolor{shadecolor}{rgb}{.97, .97, .97}
\definecolor{messagecolor}{rgb}{0, 0, 0}
\definecolor{warningcolor}{rgb}{1, 0, 1}
\definecolor{errorcolor}{rgb}{1, 0, 0}
\newenvironment{knitrout}{}{} % an empty environment to be redefined in TeX

\usepackage{alltt}
\usepackage{tikz}

% make it pretty
% get rid of junk
\usetheme{default}
\usefonttheme[onlymath]{serif}
\beamertemplatenavigationsymbolsempty

% define a bunch of colors
\definecolor{offwhite}{RGB}{255,250,240}
\definecolor{gray}{RGB}{155,155,155}
\definecolor{foreground}{RGB}{80,80,80}
\definecolor{background}{RGB}{255,255,255}
%\definecolor{title}{RGB}{255,199,0}
\definecolor{title}{RGB}{89,132,212}
%\definecolor{subtitle}{RGB}{89,132,212}
\definecolor{subtitle}{RGB}{255,199,0}
\definecolor{hilit}{RGB}{248,117,79}
\definecolor{vhilit}{RGB}{255,111,207}
\definecolor{lolit}{RGB}{200,200,200}
\definecolor{lit}{RGB}{255,199,0}
\definecolor{mdlit}{RGB}{89,132,212}
\definecolor{link}{RGB}{248,117,79}

% a few color macros
\newcommand{\hilit}{\color{hilit}}
\newcommand{\vhilit}{\color{vhilit}}
\newcommand{\lit}{\color{lit}}
\newcommand{\mdlit}{\color{mdlit}}
\newcommand{\lolit}{\color{lolit}}

% use those colors
\setbeamercolor{titlelike}{fg=title}
\setbeamercolor{subtitle}{fg=subtitle}
\setbeamercolor{frametitle}{fg=gray}
%\setbeamercolor{structure}{fg=subtitle}
\setbeamercolor{structure}{fg=title}
\setbeamercolor{institute}{fg=lolit}
\setbeamercolor{normal text}{fg=foreground,bg=background}
\setbeamertemplate{itemize subitem}{{\textendash}}
\setbeamerfont{itemize/enumerate subbody}{size=\small}
\setbeamerfont{itemize/enumerate subitem}{size=\small}

% center title of slides
\setbeamertemplate{blocks}[rounded]
\setbeamertemplate{frametitle}[default][center]

% page number
\setbeamerfont{page number in foot}{size=\footnotesize}
\setbeamertemplate{footline}[frame number]

% default link color
\hypersetup{colorlinks, urlcolor={link}}

% a few macros
\newcommand{\code}[1]{\texttt{#1}}
\newcommand{\hicode}[1]{{\hilit \texttt{#1}}}
\newcommand{\locode}[1]{{\lolit \texttt{#1}}}
\newcommand{\bb}[1]{\begin{block}{#1}}
\newcommand{\eb}{\end{block}}
\newcommand{\bi}{\begin{itemize}}
\newcommand{\bbi}{\vspace{4pt} \begin{itemize} \itemsep8pt}
\newcommand{\ei}{\end{itemize}}
\newcommand{\bv}{\begin{verbatim}}
\newcommand{\ev}{\end{verbatim}}
\newcommand{\ig}{\includegraphics}
\newcommand{\subt}[1]{{\footnotesize \color{subtitle} {#1}}}
\newcommand{\ttsm}{\tt \small}
\newcommand{\ttfn}{\tt \footnotesize}
\newcommand{\figh}[2]{\centerline{\includegraphics[height=#2\textheight]{#1}}}
\newcommand{\figw}[2]{\centerline{\includegraphics[width=#2\textwidth]{#1}}}



%------------------------------------------------

\title{Linear Regression (part 1)}
\subtitle{Intro 2 Statistical Learning}
\author{\href{http://www.gastonsanchez.com}{Gaston Sanchez}}
\institute{\href{https://creativecommons.org/licenses/by-sa/4.0/}{\tt \scriptsize \color{foreground} CC BY-SA 4.0}}
\date{}
\IfFileExists{upquote.sty}{\usepackage{upquote}}{}
\begin{document}



% no page number in first slide
{
  \setbeamertemplate{footline}{} 
  \frame{\titlepage} 
}

%------------------------------------------------

\begin{frame}
\begin{center}
\Huge{\hilit{Linear Regression}}
\end{center}
\end{frame}

%------------------------------------------------

\begin{frame}[fragile]
\frametitle{\code{Advertising} Data from ISL}

\begin{knitrout}\scriptsize
\definecolor{shadecolor}{rgb}{0.969, 0.969, 0.969}\color{fgcolor}\begin{kframe}
\begin{alltt}
\hlcom{# file in folder data/ of github repo}
\hlstd{Advertising} \hlkwb{<-} \hlkwd{read.csv}\hlstd{(}\hlstr{"data/Advertising.csv"}\hlstd{,} \hlkwc{row.names} \hlstd{=} \hlnum{1}\hlstd{)}
\end{alltt}
\end{kframe}
\end{knitrout}

\begin{knitrout}\footnotesize
\definecolor{shadecolor}{rgb}{0.969, 0.969, 0.969}\color{fgcolor}\begin{kframe}
\begin{verbatim}
       TV   Radio   Newspaper   Sales
1   230.1    37.8        69.2    22.1
2    44.5    39.3        45.1    10.4
3    17.2    45.9        69.3     9.3
4   151.5    41.3        58.5    18.5
5   180.8    10.8        58.4    12.9
6     8.7    48.9        75.0     7.2
7    57.5    32.8        23.5    11.8
8   120.2    19.6        11.6    13.2
\end{verbatim}
\end{kframe}
\end{knitrout}

{\lolit (first 8 rows)}

\end{frame}

%------------------------------------------------

\begin{frame}
\frametitle{Advertising Data from ISL}

\textbf{Advertising} consists of:

\bi
  \item the \code{Sales} of a product in 200 different markets
  \item the advertising budgets for three different media:
  \bi
    \item \code{TV}
    \item \code{Radio}
    \item \code{Newspaper}
  \ei
  \item It is not possible to directly increase the sales of the product
  \item On the other hand, it is possible to control the advertising expenditure 
  in each of the 3 media
\ei

\end{frame}

%------------------------------------------------

\begin{frame}
\frametitle{Introduction}

\bbi
  \item Suppose we observe a quantitative response $Y$ and $p$ 
  different predictors, $X_1, X_2, \dots, X_p$
  \item We assume there is some relationship between $Y$ and $[X_1, \dots, X_p]$.
  that can be written in a general form as
  $$
  Y = f(X_1, X_2, \dots, X_p) + \epsilon
  $$
  \item $f$ represents the systematic information that the predictors provide about $Y$
  \item $\epsilon$ represents an \textit{error} term that is a catch-all for what we miss with the model
\ei

\end{frame}

%------------------------------------------------

\begin{frame}[fragile]
\frametitle{Data set \code{Advertising}}

Response:
\bi
  \item  $Y$: \code{Sales}
\ei 

Predictors:
\bi
  \item $X_1$: \code{TV}
  \item $X_2$: \code{Radio}
  \item $X_3$: \code{Newspaper}
\ei

\bigskip
Relationship:
\begin{center}
\code{Sales} = $f$(\code{TV}, \code{Radio}, \code{Newspaper}) + $\epsilon$
\end{center}

\end{frame}

%------------------------------------------------

\begin{frame}
\frametitle{Introduction}

One possibility for $f()$ is a linear relationship of the form:
{\Large
$$
Y = \beta_0 + \beta_1 X_1 + \dots + \beta_p X_p + \varepsilon
$$
}

\pause
\bi
  \item It assumes a linear dependence of $Y$ on the predictors
  \item $\beta_0, \beta_1, \dots, \beta_p$ are unknown constants
  also known as the model \textit{coefficients} or \textit{parameters}
  \item The linearity is in the parameters (i.e. coefficients)
\ei

\end{frame}

%------------------------------------------------

\begin{frame}
\frametitle{Linear relationship}

$$
Y = \beta_0 + \beta_1 X_1 + \dots + \beta_p X_p + \epsilon
$$

\bigskip

\begin{center}
\code{Sales} = $\beta_0$ + $\beta_1$ \code{TV} + $\beta_2$ \code{Radio} + $\beta_3$ \code{Newspaper} + $\epsilon$
\end{center}

\end{frame}

%------------------------------------------------

\begin{frame}
\frametitle{Examples of linear models}

A couple of examples of other possible linear models

$$
Y = \beta_0 + \beta_1 X_1 + \beta_2 log(X_2) + \beta_3 X_1 X_2 + \varepsilon 
$$

$$
Y = \beta_0 + \beta_1 X_1 + \beta_2 X_{1}^{2} + \beta_3 (X_{1}^{X_2}) + \varepsilon 
$$

\end{frame}

%------------------------------------------------

\begin{frame}
\frametitle{Non-linear models}

Some models are not linear in the parameters:
$$
Y = \beta_0 + \beta_1 X_{1}^{\beta_2} + \varepsilon 
$$

Some relationships can be transformed to linearity, for example:
$$
Y = \beta_0 X_{1}^{\beta_1} \varepsilon
$$
can be linearized by taking logs (and reexpressing some of the parameters)

$$
log(Y) = log(\beta_0) + \beta_1 log(X_1) + log(\varepsilon)
$$

\end{frame}

%------------------------------------------------

\begin{frame}
\frametitle{Introduction}

The challenge involves finding parameter estimates denoted by 
$$
\hat{\beta_0}, \hat{\beta_1}, \dots, \hat{\beta_p}
$$
that provide the ``best'' approximation for $Y$:

$$
Y \approx \hat{\beta_0} + \hat{\beta_1} X_1 + \dots + \hat{\beta_p} X_p
$$

or more commonly

$$
\hat{Y} = \hat{\beta_0} + \hat{\beta_1} X_1 + \dots + \hat{\beta_p} X_p
$$

\end{frame}

%------------------------------------------------

\begin{frame}
\frametitle{Introduction}

\bbi
  \item Linearity is a BIG assumption.
  \item True regression functions are rarely linear.
  \item Although it may seem overly simplistic, linear regression is 
  extremely useful both conceptually and practically.
  \item Quoting famous statistician George Box: \\
  \textit{``All models are worng, but some are useful.''}
\ei

\end{frame}

%------------------------------------------------

\begin{frame}
\begin{center}
\Huge{\hilit{Simple Linear Regression}}
\end{center}
\end{frame}

%------------------------------------------------

\begin{frame}
\frametitle{Simple Linear Regression with one predictor}

\bbi
  \item Simple Linear Regression = Univariate regression
  \item One predictor variable $X$ and one response variable $Y$
\ei

{\large
$$
Y = \beta_0 + \beta_1 X + \varepsilon
$$
}

\end{frame}

%------------------------------------------------

\begin{frame}
\frametitle{Simple Linear Regression with one predictor}

We assume a linear model

{\large
$$
Y = \beta_0 + \beta_1 X + \varepsilon
$$
}

where:

\bi
  \item $\beta_0$ and $\beta_1$ are two unknown constants also known 
  as \textit{coefficients} or \textit{parameters}
  \item $\beta_0$ represents the \textit{intercept}
  \item $\beta_1$ represents the \textit{slope}
  \item $\varepsilon$ is the error term
\ei

\end{frame}

%------------------------------------------------

\begin{frame}
\frametitle{Simple Linear Regression with one predictor}

In vector notation:
{\large
$$
\mathbf{y} = \beta_0 + \beta_1 \mathbf{x} + \boldsymbol{\varepsilon}
$$
}

where:
\bi
  \item $\mathbf{y}$ is the vector representing the response variable 
  \item $\mathbf{x}$ is the vector representing the predictor variable 
  \item $\boldsymbol{\varepsilon}$ is the vector representing the error term
\ei

\end{frame}

%------------------------------------------------

\begin{frame}
\frametitle{Some vector-matrix notation}

In matrix notation:
$$
\underset{n\times 1}{\mathbf{y}} =  \underset{n \times 2}{\mathbf{X}} \times 
\underset{2\times 1}{\boldsymbol{\beta}} + \underset{n\times 1}{\boldsymbol{\varepsilon}}
$$

which can be represented by:
$$
\begin{bmatrix} 
y_1 \\
y_2 \\
y_3 \\
\vdots \\
y_n \\
\end{bmatrix}
=
\begin{bmatrix} 
1 & x_{1} \\
1 & x_{2} \\
1 & x_{3} \\
\vdots & \vdots \\
1 & x_{n} \\
\end{bmatrix}
\
\begin{bmatrix} 
\beta_0 \\ 
\beta_1 \\
\end{bmatrix}
+
\begin{bmatrix} 
\varepsilon_1 \\
\varepsilon_2 \\
\varepsilon_3 \\
\vdots \\
\varepsilon_n \\
\end{bmatrix}
$$

\end{frame}

%------------------------------------------------

\begin{frame}
\frametitle{Simple Linear Regression with one predictor}

Note that if the data is centered (mean = 0)
{\large
$$
\mathbf{y} = \beta_1 \mathbf{x} + \boldsymbol{\varepsilon}
$$
}

then there is no intercept term $\beta_0$

\end{frame}

%------------------------------------------------

\begin{frame}
\frametitle{Some vector-matrix notation}

With centered data we have:
$$
\underset{n\times 1}{\mathbf{y}} =  \underset{n \times 1}{\mathbf{x}} \times 
\beta_1 + \underset{n\times 1}{\boldsymbol{\varepsilon}}
$$

which can be represented by:
$$
\begin{bmatrix} 
y_1 \\
y_2 \\
y_3 \\
\vdots \\
y_n \\
\end{bmatrix}
=
\begin{bmatrix} 
x_{1} \\
x_{2} \\
x_{3} \\
\vdots \\
x_{n} \\
\end{bmatrix}
\
\begin{bmatrix} 
\beta_1 \\
\end{bmatrix}
+
\begin{bmatrix} 
\varepsilon_1 \\
\varepsilon_2 \\
\varepsilon_3 \\
\vdots \\
\varepsilon_n \\
\end{bmatrix}
$$

\end{frame}

%------------------------------------------------

\begin{frame}
\frametitle{Various simple regressions}
\begin{center}
\ig[width=11cm]{images/advertising-regressions.png}
\end{center}
\end{frame}

%------------------------------------------------

\begin{frame}
\frametitle{Simple Linear Regression with one predictor}

Assuming the model

$$
\mathbf{y} = \beta_0 + \beta_1 \mathbf{x} + \boldsymbol{\epsilon}
$$

and given some estimates $\hat{\beta_0}$ and $\hat{\beta_1}$ for the model 
coefficients, we predict future sales using

$$
\mathbf{\hat{y}} = \hat{\beta_0} + \hat{\beta_1} \mathbf{x}
$$

where $\mathbf{\hat{y}}$ indicates the predicted $\mathbf{y}$

\end{frame}

%------------------------------------------------

\begin{frame}
\begin{center}
\Huge{\hilit{Fitting a Line}}
\end{center}
\end{frame}

%------------------------------------------------

\begin{frame}
\frametitle{Fitting a Line}
\begin{center}
\ig[width=9cm]{images/which-line1.pdf}
\end{center}
\end{frame}

%------------------------------------------------

\begin{frame}
\frametitle{Fitting a Line}
\begin{center}
\ig[width=9cm]{images/which-line2.pdf}
\end{center}
\end{frame}

%------------------------------------------------

\begin{frame}
\begin{center}
{\Large How to find the ``best'' \\ fitting line?}
\end{center}
\end{frame}

%------------------------------------------------

\begin{frame}
\frametitle{Fitting a Line}
\begin{center}
\ig[width=11cm]{images/fit-line1.pdf}
\end{center}
\end{frame}

%------------------------------------------------

\begin{frame}
\frametitle{Fitting a Line}
\begin{center}
\ig[width=11cm]{images/fit-line2.pdf}
\end{center}
\end{frame}

%------------------------------------------------

\begin{frame}
\frametitle{Fitting a Line}
\begin{center}
\ig[width=11cm]{images/fit-line3.pdf}
\end{center}
\end{frame}

%------------------------------------------------

\begin{frame}
\frametitle{Fitting a Line}
\begin{center}
\ig[width=11cm]{images/fit-line4.pdf}
\end{center}
\end{frame}

%------------------------------------------------

\begin{frame}
\frametitle{Which Criterion?}
\begin{center}
\ig[width=10cm]{images/which-criteria.pdf}
\end{center}
\end{frame}

%------------------------------------------------

\begin{frame}
\frametitle{Fitting a Line}
\begin{center}
\ig[width=11cm]{images/fit-line5.pdf}
\end{center}
\end{frame}

%------------------------------------------------

\begin{frame}
\begin{center}
\Huge{\hilit{Estimation of Parameters}}
\end{center}
\end{frame}

%------------------------------------------------

\begin{frame}
\frametitle{Estimation of the parameters}

\bi
  \item Let $\hat{y}_i = \hat{\beta_0} + \hat{\beta_1} x_i$ be the prediction
  for $\mathbf{y}$ based on the $i$th value of $\mathbf{x}$
  \pause
  \item Then $e_i = y_i - \hat{y}_i$ represents the $i$th {\hilit residual} 
  \pause
  \item We define the {\hilit Residual Sum of Squares} (RSS) as
  $$
  \text{RSS} = e_1^2 + e_2^2 + \dots + e_n^2
  $$
  \item The \textbf{Least Squares} approach chooses $\hat{\beta_0}$ and 
$\hat{\beta_1}$ to minimize the RSS
\ei

\end{frame}

%------------------------------------------------

\begin{frame}
\frametitle{Estimation of the parameters}

The starting point is to write the model as:
$$
\mathbf{e} = \mathbf{y} - (\beta_0 + \beta_1 \mathbf{x})
$$

For convenience we define a quadratic loss function
$$
L = \sum_{i=1}^{n} e_{i}^{2} = \sum_{i=1}^{n} (y_i - \beta_0 - \beta_1 x_i)^2
$$

To minimize $L$ we take partial derivatives with respect to each of the two parameters

\end{frame}

%------------------------------------------------

\begin{frame}
\frametitle{Estimation of the parameters}

Taking partial derivatives w.r.t each of the two parameters:
$$
\frac{\partial L}{\partial \beta_0} = 2 \sum_{i=1}^{n} (y_i - \beta_0 - \beta_1 x_i) (-1) = 0
$$

and
$$
\frac{\partial L}{\partial \beta_1} = 2 \sum_{i=1}^{n} (y_i - \beta_0 - \beta_1 x_i) (-x_i) = 0
$$

\end{frame}

%------------------------------------------------

\begin{frame}
\frametitle{Estimation of the parameters}

The solutions for $\beta_0$ and $\beta_1$ would be ontained by solving 
the so-called \textit{normal equations}
$$
\sum_{i=1}^{n} (y_i - \beta_0 - \beta_1 x_i) = 0
$$

and
$$
\sum_{i=1}^{n} (x_i y_i - x_i \beta_0 - \beta_1 x_{i}^{2}) = 0
$$

\end{frame}

%------------------------------------------------

\begin{frame}
\frametitle{Estimation of the parameters by OLS}

The \textbf{Least Squares} coefficients are:

\begin{align*}
\hat{\beta_1} &= \frac{\sum_{i=1}^{n} (x_i - \bar{x})(y_i - \bar{y})}{\sum_{i=1}^{n} (x_i - \bar{x})^2} \\
\vspace{5mm} \\
\hat{\beta_0} &= \bar{y} - \hat{\beta_1} \bar{x} 
\end{align*}

where:

$$
\bar{y} = \frac{1}{n} \sum_{i=1}^{n} y_i \quad \text{and} 
\quad \bar{x} = \frac{1}{n} \sum_{i=1}^{n} x_i
$$

\end{frame}

%------------------------------------------------

\begin{frame}
\frametitle{Estimation of the parameters by OLS}

Notice that:
$$
\hat{\beta_1} = \frac{\sum_{i=1}^{n} (x_i - \bar{x})(y_i - \bar{y})}{\sum_{i=1}^{n} (x_i - \bar{x})^2}
$$

is equivalent to:
$$
\hat{\beta_1} = \frac{cov(\mathbf{x}, \mathbf{y})}{var(\mathbf{x})}
$$

\end{frame}

%------------------------------------------------

\begin{frame}[fragile]
\frametitle{Example: Advertising Data}

\begin{knitrout}\footnotesize
\definecolor{shadecolor}{rgb}{0.969, 0.969, 0.969}\color{fgcolor}\begin{kframe}
\begin{alltt}
\hlcom{# number of observations}
\hlstd{n} \hlkwb{<-} \hlkwd{nrow}\hlstd{(Advertising)}

\hlcom{# model matrix}
\hlstd{x} \hlkwb{<-} \hlstd{Advertising}\hlopt{$}\hlstd{TV}

\hlcom{# reponse variable}
\hlstd{y} \hlkwb{<-} \hlstd{Advertising}\hlopt{$}\hlstd{Sales}
\end{alltt}
\end{kframe}
\end{knitrout}

\end{frame}

%------------------------------------------------

\begin{frame}[fragile]
\frametitle{Example: Advertising Data}

\begin{knitrout}\footnotesize
\definecolor{shadecolor}{rgb}{0.969, 0.969, 0.969}\color{fgcolor}\begin{kframe}
\begin{alltt}
\hlcom{# slope}
\hlstd{b1} \hlkwb{<-} \hlkwd{cov}\hlstd{(x, y)} \hlopt{/} \hlkwd{var}\hlstd{(x)}
\hlstd{b1}
\end{alltt}
\begin{verbatim}
## [1] 0.04753664
\end{verbatim}
\begin{alltt}
\hlcom{# intercept}
\hlstd{b0} \hlkwb{<-} \hlkwd{mean}\hlstd{(y)} \hlopt{-} \hlstd{b1} \hlopt{*} \hlkwd{mean}\hlstd{(x)}
\hlstd{b0}
\end{alltt}
\begin{verbatim}
## [1] 7.032594
\end{verbatim}
\end{kframe}
\end{knitrout}

\end{frame}

%------------------------------------------------

\begin{frame}
\frametitle{Example: Advertising Data}
\begin{center}
\ig[width=9cm]{images/advertising-regression-tv-sales.png}
\end{center}

{\footnotesize The least squares fit for the regression of \code{Sales} on \code{TV}. \\
In this case a linear fit captures the essence of the relationship, although 
it is somewhat deficient in the left of the plot.}

\end{frame}

%------------------------------------------------

\begin{frame}
\begin{center}
\Huge{\hilit{Another perspective}}
\end{center}
\end{frame}

%------------------------------------------------

\begin{frame}
\frametitle{Projection}

Notice that:
$$
\hat{\beta_1} = \frac{\sum_{i=1}^{n} (x_i - \bar{x})(y_i - \bar{y})}{\sum_{i=1}^{n} (x_i - \bar{x})^2}
$$

\bigskip
Can be expressed in vector notation as:
{\large
$$
\hat{\beta_1} = \frac{\mathbf{y^\mathsf{T} x}}{\mathbf{x^\mathsf{T} x}}
$$
}
with $\mathbf{x}$ and $\mathbf{y}$ mean-centered.

\end{frame}

%------------------------------------------------

\begin{frame}
\frametitle{Projection}

Thus, with centered variables $\mathbf{x}$ and $\mathbf{y}$, the fitted values 
$\mathbf{\hat{y}}$ are given by:

{\large
\begin{align*}
\mathbf{\hat{y}} &= \hat{\beta_1} \mathbf{x} \\
&= \left( \frac{\mathbf{y^\mathsf{T} x}}{\mathbf{x^\mathsf{T} x}} \right) \mathbf{x} \\
&= \mathbf{x} \left( \frac{\mathbf{y^\mathsf{T} x}}{\mathbf{x^\mathsf{T} x}} \right) \\
&= \mathbf{x} (\mathbf{x^\mathsf{T} x})^{-1} \mathbf{x^\mathsf{T} y} 
\end{align*}
}

\end{frame}

%------------------------------------------------

\begin{frame}
\frametitle{From variables perspective}
\begin{center}
\ig[width=9cm]{images/simple-ols-geometry1.pdf}
\end{center}
\end{frame}

%------------------------------------------------

\begin{frame}
\frametitle{From variables perspective}
\begin{center}
\ig[width=9cm]{images/simple-ols-geometry2.pdf}
\end{center}
\end{frame}

%------------------------------------------------

\begin{frame}
\frametitle{From variables perspective}
\begin{center}
\ig[width=9cm]{images/simple-ols-geometry3.pdf}
\end{center}
\end{frame}

%------------------------------------------------

\begin{frame}
\frametitle{Some Remarks}

\bbi
  \item There is nothing in the Least Squares method that requires statistical inference:
formal tests of null hypotheses or confidence intervals.
  \item In its simplest form, regression analysis can be performed without statistical inference.
  \item The inferential part can sometimes be very useful but goes beyond the definition of a regression analysis.
\ei

\end{frame}

%------------------------------------------------

\begin{frame}
\frametitle{Some Comments}

\bbi
  \item Linear Regression is a ``simple'' approach to supervised learning.
  \item Don't get fooled by the word ``simple''.
  \item ``simple'' $\neq$ easy / boring / uninteresting.
  \item I will use the terms \textit{Regression Analysis} and \textit{Regression Model} interchangeably.
\ei

\end{frame}

%------------------------------------------------

\begin{frame}
\frametitle{Bibliography}

{\footnotesize
\bi
  \item \textbf{Statistical Regression and Classification} by Norman Matloff (2017). CRC Press.
  \item \textbf{Linear Models with R} by Julian Faraway (2015). CRC Press.
  \item \textbf{Modern Regression Methods} by Thomas Ryan (2009). Wiley.
  \item \textbf{A Modern Approach to Regression with R} by Simon Sheather (2009). Springer.
  \item \textbf{Modern Multivariate Statistical Techniques} by Julian Izenman (2008). Springer.
  \item \textbf{Data Mining and Statistics for Decision Making} by Stephane Tuffery (2011).
  \textit{Chapter 11: Classification and prediction methods}. Wiley.
\ei
}

\end{frame}

%------------------------------------------------

\begin{frame}
\frametitle{French Literature}

{\footnotesize
\bi
  \item \textbf{Probabilites, analyse des donnees et statistique} by Gilbert Saporta (2011).
  \textit{Chapter 17: La regression multiple et le modele lineaire general}. 
  Editions Technip, Paris.
  \item \textbf{Statistique: Methodes pour decrire, expliquer et prevoir} 
  by Michel Tenenhaus (2008). \textit{Chapter 5: La Regression Multiple}. Dunod, Paris.
  \item \textbf{Regression avec R} by Cornillon and Matzner-Lober (2011). Springer.
  \item \textbf{Statistique Exploratoire Multidimensionnelle} by Lebart et al (2004).
  \textit{Chapter 3, section 3.2: Regression multiple, modele lineaire}. Dunod, Paris.
  \item \textbf{Traitement des donnees statistiques} by Lebart et al. (1982). 
  \textit{Unit 3: Modele Lineaire}. Dunod, Paris.
\ei
}

\end{frame}

%------------------------------------------------

\end{document}
